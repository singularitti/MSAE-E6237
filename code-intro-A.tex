% !TEX root = report.tex

\section{Molecular dynamics code}

The following part is an introduction on how to use
Wentzcovitch's code to perform MD simulation. The code is
based on one of her papers.\cite{Wentzcovitch:1991ka}

There are $4$ types of simulations in code,
denoted as \texttt{md}, \texttt{cd}, \texttt{nd}, \texttt{sd},
respectively. The first one means ``molecular dynamics'',
based on Andersen's paper,\cite{Andersen:1980ew},
the second one means ``cell dynamics'', based on
Rahman–Parrinello approach,\cite{Parrinello:1980kx},
the main equations are $\eqref{eq:rpeqm}$.
The third case is called ``new cell dynamics'' and is based on
$\eqref{eq:rpsdd}$ and $\eqref{eq:wenzhdd2}$,
the last one is called ``strain dynamics'' and is based on
$\eqref{eq:rpsdd}$ and $\eqref{eq:wenzhdd}$.

The simulation is done under zero temperature, as stated above.

\texttt{mxdtyp} denotes the array dimension for type of atoms,
in the input file there will be a line labeled by \texttt{(ntype)}, it
contains a scalar, i.e., number of atom types, so \texttt{mxdtyp} $=1$.

\subsection{Main program}

Then we start a MD loop.
First update the current step \texttt{nzero}, and then update those
accumulators \texttt{acu}, \texttt{ack} and \texttt{acp}, then calculate
their averages by dividing \texttt{nzero}. Then call \texttt{move} subroutine,
\texttt{p} is updated during this step, however \texttt{vcell} is not.
Then calculate $p V$ and do a corresponding accumulator \texttt{acpv} and
average \texttt{avpv}. And then, update arrays like \texttt{utm} with length
\texttt{nstep} (read from input). 
Then calculate the new temperature, \texttt{tnew}, by assuming equipartition 
theorem $\bar{E} = \frac{ 3 }{ 2 } k_B T$, where $\bar{E}$ is
the average kinetic energy \texttt{avk}.
Then do rescaling on Cartesian velocity \texttt{v} and reduced velocity
\texttt{ratd}.

At the end of MD loop, write energy outputs to file \texttt{e}: total potential energy $U_t$ as 
\texttt{utm} array, total kinetic energy $E_t$ as \texttt{ekintm} array,
total energy $\mathscr{E}_t = E_t + U_t$ as \texttt{etotm} array, and
total $p \Omega$ as \texttt{pvm}. And, write decomposed energy outputs
to file \texttt{eal}: atomic contribution to total potential energy as 
\texttt{utam} array, to total kinetic energy as \texttt{ekam} array,
to total energy as \texttt{etam} array; and lattice contribution to
total potential energy as \texttt{utlm} array, to total kinetic 
energy as \texttt{eklm} array,
to total energy as \texttt{etlm} array. It could be also understood as
ionic and cell contributions, respectively. Then write average accumulated
outputs to file \texttt{ave}, by \texttt{avum}, \texttt{avkm}, etc. Another
important output file is \texttt{avec}, which stores lengths and angles 
between primitive cell vectors for each MD step, 
denoted by \texttt{bmodm} and
\texttt{thetam} matrices. \texttt{bmodm} at each MD step
is set to lattice vectors moduli \texttt{avmod}, computed by \texttt{MOVE}.
\texttt{tv} file stores the ``instantaneous'' temperature and volume
for each MD step.

% Table generated by Excel2LaTeX from sheet 'Sheet1'
\begin{table}[h]
	\centering
	\caption{List of variables.}
	\begin{tabular}{ccccc}
		\multicolumn{2}{c}{variable name} & variable symbol & variable & write to file \\
		\multirow{4}[0]{*}{total}                         & potential energy     & $U_t$           & \texttt{utm}    & \multirow{4}[0]{*}{\texttt{e}}   \\
		                                                  & total kinetic energy & $E_t$           & \texttt{ekintm} &                                  \\
		                                                  & energy               & $\mathscr{E}_t$ & \texttt{etotm}  &                                  \\
		                                                  & $p \Omega$           & $p \Omega$      & \texttt{pvm}    &                                  \\
		\multirow{3}[0]{*}{atomic contribution to total}  & potential energy     &                 & \texttt{utam}   & \multirow{6}[0]{*}{\texttt{eal}} \\
		                                                  & kinetic energy       &                 & \texttt{ekam}   &                                  \\
		                                                  & energy               &                 & \texttt{etam}   &                                  \\
		\multirow{3}[0]{*}{lattice contribution to total} & potential energy     &                 & \texttt{utlm}   &                                  \\
		                                                  & kinetic energy       &                 & \texttt{eklm}   &                                  \\
		                                                  & energy               &                 & \texttt{etlm}   &                                  \\
		\multirow{2}[0]{*}{average of accumulated}        & potential energy     & $\bar{U}_t$     & \texttt{avum}   & \multirow{2}[0]{*}{\texttt{ave}} \\
		                                                  & kinetic energy       & $\bar{E}_t$     & \texttt{avkm}   &                                  \\
	\end{tabular}%
	\label{tab:lstvar}%
\end{table}%



\subsection{Setup steps}

\subsubsection{\texttt{CRSTL} subroutine}

This subroutine does some pre-setting work before \texttt{INIT}.
The \texttt{avec} parameter is the $3$ lattice primitive vectors in Cartesian
coordinates,
i.e., $\{ \bm{a}, \bm{b}, \bm{c} \}$,
denoted as $h$ in Wentzcovitch's paper.\cite{Wentzcovitch:1991ka}
\texttt{rat} is the atomic positions in terms of lattice primitive vectors,
\texttt{ratd} is its first-order time derivative.
\texttt{g} is just $g = h \tran h$ and \texttt{gm1} is $g^{-1}$.
\texttt{cmass} is the fictitious mass $W$ and
\texttt{press} is the external pressure $P​$ in
Wentzcovitch's paper.\cite{Wentzcovitch:1991ka}


\subsubsection{\texttt{INIT} subroutine}

First, call \texttt{RANV} subroutine, and initialize \texttt{avecd},
\texttt{avec2d} and \texttt{gd}, etc.
\texttt{ilj} is the only variable set by hand in code, if it is equal to $1$,
the code will call \texttt{FORCLJ} subroutine, else it will call \texttt{FORC}
subroutine, see \ref{sssec:forc} and \ref{sssec:forclj} for differences.

Then if \texttt{calc} flag is not set to \texttt{md} and \texttt{mm}
then \texttt{gmgd} is calculated.

If \texttt{calc} flag is set to \texttt{nd} or \texttt{nm}
then \texttt{SIGP} is called, if it is set to \texttt{sd} or
\texttt{sm} then \texttt{SIGS} is called, see \ref{sssec:sigs&p}
for more detail.
With
$\sigma_0 = 
\{
\bm{a}_0 \times \bm{b}_0, \bm{b}_0 \times \bm{c}_0,
\bm{c}_0 \times \bm{a}_0 
\}
= \frac{ V_0 }{ 2\pi } \{
\bm{c}^\ast_0, \bm{a}^\ast_0, \bm{b}^\ast_0
\}$,
we know
$\ddot{d} = \frac{ 1 }{ V_0 }\ddot{h} \sigma_0$.
Then we do strain symmetrization, $\ddot{d}_{ij} = \frac{ 1 }{ 2 }
(\ddot{d}_{ij} + \ddot{d}_{ji})$.
Then $\ddot{h} = \ddot{d} h_0$.

If in \texttt{nd} or \texttt{nm}, we do 
$\Tr (\dot{h}\tran \sigma \sigma \dot{h})$.

If in \texttt{sd} or \texttt{sm}, we do 
$\Tr(\dot{h}\tran \sigma_0 \sigma_0\tran \dot{ h })$.

Then in both cases, we do
\texttt{ekl} $=$ \texttt{ekl} + $\Tr (\dot{ h }\tran \dot{ h })$.

Total kinetic energy is $E_{t} = $ \texttt{eka} + $\frac{ 1 }{ 2 } W ekl$,
where $W$ is the fictitious mass,
total potential energy $U_{t} = ???E_{t} + P \Omega$, where $P$ is the external pressure.
Total energy is $\mathscr{E}_{t} = E_t + U_{t}$.


\subsubsection{\texttt{RDPP} subroutine}

This subroutine reads the pair-potential file.
Here \texttt{ntype} is the number of different types of atoms.
It reads the potential of $i$th atom and $j$th atom, where $j \geq i$.
So in the code we need to do some assignments like
\begin{align}
	U_{ij}      & = U_{ji},      \\
	\bm{F}_{ij} & = \bm{F}_{ji}, 
\end{align}
where $U_{ji}$ and $\bm{F}_{ji}$ are what we read from file,
thus we can save time on IO operations.


\subsubsection{\texttt{RANV} subroutine}

This subroutine is dedicated for setting up Maxwell distributed random velocities
at temperature $T$.


