% !TEX root = report.tex

\section{Molecular dynamics code}

The following part is an introduction on how to use
Wentzcovitch's code to perform MD simulation. The code is
based on one of her papers.\cite{Wentzcovitch:1991ka}

There are $4$ types of simulations in code,
denoted as \texttt{md}, \texttt{cd}, \texttt{nd}, \texttt{sd},
respectively. The first one means ``molecular dynamics'',
based on Andersen's paper,\cite{Andersen:1980ew},
the second one means ``cell dynamics'', based on
Rahman–Parrinello approach,\cite{Parrinello:1980kx},
the main equations are $\eqref{eq:rpeqm}$.
The third case is called ``new cell dynamics'' and is based on
$\eqref{eq:rpsdd}$ and $\eqref{eq:wenzhdd2}$,
the last one is called ``strain dynamics'' and is based on
$\eqref{eq:rpsdd}$ and $\eqref{eq:wenzhdd}$.

The simulation is done under zero temperature, as stated above.

\texttt{mxdtyp} denotes the array dimension for type of atoms,
in the input file there will be a line labeled by \texttt{(ntype)}, it
contains a scalar, i.e., number of atom types, so \texttt{mxdtyp} $=1$.

\subsection{Setup steps}

\subsubsection{\texttt{CRSTL} subroutine}

This subroutine does some pre-setting work before \texttt{INIT}.
The \texttt{avec} parameter is the $3$ lattice primitive vectors in Cartesian
coordinates,
i.e., $\{ \bm{a}, \bm{b}, \bm{c} \}$,
denoted as $h$ in Wentzcovitch's paper.\cite{Wentzcovitch:1991ka}
\texttt{rat} is the atomic positions in terms of lattice primitive vectors,
\texttt{ratd} is its first-order time derivative.
\texttt{g} is just $g = h \tran h$ and \texttt{gm1} is $g^{-1}$.
\texttt{cmass} is the fictitious mass $W$ and
\texttt{press} is the external pressure $P​$ in
Wentzcovitch's paper.\cite{Wentzcovitch:1991ka}


\subsubsection{\texttt{INIT} subroutine}

First, call \texttt{RANV} subroutine, and initialize \texttt{avecd},
\texttt{avec2d} and \texttt{gd}, etc.
\texttt{ilj} is the only variable set by hand in code, if it is equal to $1$,
the code will call \texttt{FORCLJ} subroutine, else it will call \texttt{FORC}
subroutine, see \ref{sssec:forc} and \ref{sssec:forclj} for differences.

Then if \texttt{calc} flag is not set to \texttt{md} and \texttt{mm}
then \texttt{gmgd} is calculated.

If \texttt{calc} flag is set to \texttt{nd} or \texttt{nm}
then \texttt{SIGP} is called, if it is set to \texttt{sd} or
\texttt{sm} then \texttt{SIGS} is called, see \ref{sssec:sigs&p}
for more detail.
Then we do strain symmetrization, $




\subsubsection{\texttt{RDPP} subroutine}

This subroutine reads the pair-potential file.
Here \texttt{ntype} is the number of different types of atoms.
It reads the potential of $i$th atom and $j$th atom, where $j \geq i$.
So in the code we need to do some assignments like
\begin{align}
	U_{ij}      & = U_{ji},      \\
	\bm{F}_{ij} & = \bm{F}_{ji},
\end{align}
where $U_{ji}$ and $\bm{F}_{ji}$ are what we read from file,
thus we can save time on IO operations.


\subsubsection{\texttt{RANV} subroutine}

This subroutine is dedicated for setting up Maxwell distributed random velocities
at temperature $T$.


