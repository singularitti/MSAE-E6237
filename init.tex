% !TEX root = ./main.tex

\subsubsection{\texttt{INIT} subroutine}
\label{sssec:init}

This subroutine does some initialization work for later calculations.
First call \texttt{RANV} subroutine,
generate some initial random positions for each atom of each type,
see Sec. \ref{sssec:ranv}.
\texttt{ilj} is the only variable set by hand in code, if it is equal to $1$,
the code will call \texttt{FORCLJ} subroutine, otherwise it will call \texttt{FORC}
subroutine, see Sec. \ref{sssec:forc} and Sec. \ref{sssec:forclj} for their differences.

Then if \texttt{calc} flag is not set to \texttt{md} and \texttt{mm},
Rahman--Parrinello approach is conducted, see Sec. \ref{sssec:rpa} for detail.
At the beginning, \texttt{gmgd}, i.e., $g^{-1} \dot{g}$ is calculated.
Then for each atom of each type,
\texttt{rat2d}, i.e., $\ddot{\bm{s}}_i = g^{-1} \dot{g} \dot{ \bm{s} }_i$ is calculated.
Also, \texttt{pim}, the $\Pi$ matrix is calculated according to $\eqref{eq:pim&frr}$.
And $\ddot{h}$ is calculated according to $\eqref{eq:hdd}$.

If \texttt{calc} flag is set to \texttt{nd} or \texttt{nm}
then \texttt{SIGP} is called, if it is set to \texttt{sd} or
\texttt{sm} then \texttt{SIGS} is called, see Sec. \ref{sssec:sigs&p}
for more details.
From these $2$ subroutines we know
\begin{equation}
	\ddot{d} = \frac{ \ddot{h} }{ h_0 } = \frac{ \ddot{h} \sigma_0}{V_0},
\end{equation}
where
$\sigma_0 =
	\{
	\bm{b}_0 \times \bm{c}_0, \bm{c}_0 \times \bm{a}_0,
	\bm{a}_0 \times \bm{b}_0
	\}
	= \frac{ V_0 }{ 2\pi } \{
	\bm{a}^\ast_0, \bm{b}^\ast_0, \bm{c}^\ast_0
	\} = \frac{ V_0 }{ \{\bm{a}_0, \bm{b}_0, \bm{c}_0 \} } = \frac{ V_0 }{ h_0 }$.
However, $\ddot{h}$ is not symmetrized here, so we need to do
strain symmetrization, $\ddot{d}_{ij} = \frac{ 1 }{ 2 }
	(\ddot{d}_{ij} + \ddot{d}_{ji})$, and then solve
$\ddot{h} = \ddot{d} h_0$.

The lattice contribution to kinetic energy is
\begin{equation}\label{eq:calc}
	e_l =
	\begin{cases}
		\Tr (\dot{h}\tran \sigma \sigma\tran \dot{h}),      & \text{\texttt{calc} is \texttt{nd} or \texttt{nm};} \\
		\Tr(\dot{h}\tran \sigma_0 \sigma_0\tran \dot{ h }), & \text{\texttt{calc} is \texttt{sd} or \texttt{sm};} \\
		\Tr(\dot{h}\tran \dot{ h }),                        & \text{\texttt{calc} is \texttt{md} or \texttt{mm}.}
	\end{cases}
\end{equation}

Then it calculates initial energies.
Total kinetic energy is $E_t = \sum_{i} \frac{ 1 }{ 2 } m_i v_i^2 + \frac{ 1 }{ 2 } W e_l$,
where $W$ is the fictitious mass, and $\sum_{i} \frac{ 1 }{ 2 } m_i v_i^2$
results from \texttt{RANV} subroutine.
Total potential energy $U_{t} = \sum_{i} u_i + P \Omega$, where $P$ is the external pressure,
and $\sum_{i} u_i$ results from \texttt{FORCLJ} or \texttt{FORC} subroutine.
Total energy is $\mathscr{E}_{t} = E_t + U_{t}$.

\begin{table}[h]
	\centering
	\caption{List of some variables in \texttt{INIT} subroutine.}
	\begin{tabular}{@{}lcr@{}}
		\toprule
		variable name           & variable symbol                       & variable        \\
		\midrule
		$\Pi$ matrix            & $\Pi$                                 & \texttt{pim}    \\
		forces on basis vectors & $\ddot{h}$                            & \texttt{avec2d} \\
		pressure                & $P$                                   & \texttt{press}  \\
		                        & $d = (1 + \epsilon)\tran(1+\epsilon)$ & \texttt{d2}     \\
		\bottomrule
	\end{tabular}%
	\label{tab:init}%
\end{table}%
