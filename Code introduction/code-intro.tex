% !TEX root = report.tex

\section{Molecular dynamics code}

The following part is an introduction on how to use
Wentzcovitch's code to perform MD simulation. The code is
based on one of her papers.\cite{Wentzcovitch:1991ka}

There are $4$ types of simulations in code,
denoted as \texttt{md}, \texttt{cd}, \texttt{nd}, \texttt{sd},
respectively. The first one means ``molecular dynamics'',
based on Andersen's paper,\cite{Andersen:1980ew},
the second one means ``cell dynamics'', based on
Rahman–Parrinello approach,\cite{Parrinello:1980kx},
the main equations are $\eqref{eq:rpeqm}$.
The third case is called ``new cell dynamics'' and is based on
$\eqref{eq:rpsdd}$ and $\eqref{eq:wenzhdd2}$,
the last one is called ``strain dynamics'' and is based on
$\eqref{eq:rpsdd}$ and $\eqref{eq:wenzhdd}$.

The simulation is done under zero temperature, as stated above.

\texttt{mxdtyp} denotes the array dimension for type of atoms,
in the input file there will be a line labeled by \texttt{(ntype)}, it
contains a scalar, i.e., number of atom types, so \texttt{mxdtyp} $=1$.

\input{"Code introduction/celq"}

\subsection{Setup steps}

\input{"Code introduction/crstl"}

\input{"Code introduction/init"}

\input{"Code introduction/rdpp"}

\input{"Code introduction/ranv"}

\subsection{Force calculation}

\input{"Code introduction/forclj"}

\input{"Code introduction/forc"}

\input{"Code introduction/updg"}

\input{"Code introduction/sigsp"}
