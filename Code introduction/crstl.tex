% !TEX root = ../report.tex

\subsubsection{\texttt{CRSTL} subroutine}
\label{sssec:crstl}

This subroutine does some pre-setting work before \texttt{INIT}.

First read the $3 \times 3$ matrix \texttt{avec} from \texttt{inp} file.
If \texttt{ic} flag is \texttt{'s'},
since it is in reduced coordinates, we need to rewrite it in Cartesian
coordinates. First store the sign of each entry, the do some transformation according to
the corresponding \texttt{iop} matrix entry(\textcolor{red}{What is this?}). The
rules are
\begin{equation}\label{eq:crstltrans}
	h(i, j) = \sign \big( h(i, j) \big) \times a_0 \times n(j)
	\begin{cases}
		\sqrt{h(i, j)},                 & \text{\texttt{iop(i,j)} is \texttt{'s'};} \\
		h(i, j)^{1/3},                  & \text{\texttt{iop(i,j)} is \texttt{'c'};} \\
		\frac{ 1 }{ 3 } h(i, j),        & \text{\texttt{iop(i,j)} is \texttt{'t'};} \\
		\frac{ 1 }{ \sqrt{3} } h(i, j), & \text{\texttt{iop(i,j)} is \texttt{'h'},}
	\end{cases}
\end{equation}
where $n(j)$ denotes the number of primitive cells along $j$th direction.
If \texttt{ic} is \texttt{'c'}, i.e., in an intermediate run,
then read \texttt{avec}, \texttt{avecd}, \texttt{avec2di} from last run as
input for subsequent runs, and write them to standard output.

\begin{table}[h]
	\centering
	\caption{List of some variables in \texttt{CRSTL} subroutine.}
	\begin{tabular}{@{}lcr@{}}
		\toprule
		variable name                          & variable symbol                                 & variable       \\
		\midrule
		lattice vectors matrix                 & $h = \{ \bm{a}, \bm{b}, \bm{c} \}$              & \texttt{avec}  \\
		atomic position in reduced coordinates & $\bm{s}_i = \{\xi_i, \eta_i, \zeta_i\}$         & \texttt{rat}   \\
		atomic velocity in reduced coordinates & $\dot{ \bm{s} }_i$                              & \texttt{ratd}  \\
		number of primitive cells              & $n$                                             & \texttt{nsc}   \\
		lattice parameter                      & $a_0$                                           & \texttt{alatt} \\
		fictitious mass                        & $W$                                             & \texttt{cmass} \\
		external pressure                      & $P$                                             & \texttt{press} \\
		reciprocal lattice vectors matrix      & $\sigma$                                        & \texttt{sigma} \\
		metric tensor                          & $g = h \tran h$                                 & \texttt{g}     \\
		metric tensor velocity                 & $\dot{g} = \dot{ h }\tran h + h \dot{ h }\tran$ & \texttt{gd}    \\
		metric tensor inverse                  & $g^{-1}$                                        & \texttt{gm1}   \\
		\bottomrule
	\end{tabular}%
	\label{tab:crstl}%
\end{table}%

Then calculate cell volume $\Omega$, metric tensor $g$ and $g^{-1}$, which are
treated as part of the outputs of this subroutine.

Then read positions \texttt{rat} for each atom of each type.
If \texttt{ic} flag is \texttt{'s'},
then do similar transformation of these positions as $\eqref{eq:crstltrans}$ does;
if it is \texttt{'c'}, i.e., it is an intermediate step, then read
\texttt{avec}, \texttt{avecd}, \texttt{avec2di} from last run,
and write to standard output.

Then it calls \texttt{SPCEL} subroutine, build super cell along $3$ directions.

Read \texttt{nstep} from input file, determine how many steps are
to be performed in the following MD loop.

At last it calls \texttt{RDPP} subroutine.
