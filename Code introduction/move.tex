% !TEX root = ../report.tex

\subsubsection{\texttt{MOVE} subroutine}

Then it calculates total atomic kinetic energy by
$E_a = \frac{ 1 }{ 2 } \sum_{i=1}^{N} m_i \dot{ \bm{s} }_i \tran
g \dot{ \bm{s} }_i$ for each atom of each type,
and total atomic potential $U_a = \sum_{i} u_i$ from \texttt{FORCLJ} or \texttt{FORC}
subroutines.
When it comes to the lattice contribution, it is the same as \texttt{INIT} subroutine,
i.e., as $\eqref{eq:calc}$ shows. Then lattice contribution $U_l$, $E_l$, and
$\mathscr{E}_l$ can be calculated.

\begin{table}[h]
	\centering
	\caption{List of some variables in \texttt{MOVE} subroutine.}
	\begin{tabular}{@{}rlcc@{}}
		\toprule
		\multicolumn{2}{c}{variable name} & variable symbol & variable \\
		\midrule
		\multirow{3}[2]{*}{atomic contribution to total}                                  & potential energy & $U_a$           & \texttt{uta} \\
		                                                                                  & kinetic energy   & $E_a$           & \texttt{eka} \\
		                                                                                  & energy           & $\mathscr{E}_a$ & \texttt{eta} \\
		\cmidrule{1-1}\cmidrule{4-4}    \multirow{3}[2]{*}{lattice contribution to total} & potential energy & $U_l$           & \texttt{utl} \\
		                                                                                  & kinetic energy   & $E_l$           & \texttt{ekl} \\
		                                                                                  & energy           & $\mathscr{E}_l$ & \texttt{etl} \\
		\bottomrule
	\end{tabular}
	\label{tab:move}%
\end{table}%
