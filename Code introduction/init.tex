% !TEX root = report.tex

\subsubsection{\texttt{INIT} subroutine}

First call \texttt{RANV} subroutine,
generate some initial random positions for each atom of each type.
\texttt{ilj} is the only variable set by hand in code, if it is equal to $1$,
the code will call \texttt{FORCLJ} subroutine, else it will call \texttt{FORC}
subroutine, see \ref{sssec:forc} and \ref{sssec:forclj} for differences.

Then if \texttt{calc} flag is not set to \texttt{md} and \texttt{mm},
Rahman--Parrinello approach is conducted, see \ref{ssec:rpa} for detail.
First, \texttt{gmgd}, i.e., $g^{-1} \dot{g}$ is calculated.
Then for each atom of each type,
\texttt{rat2d}, i.e., $\ddot{\bm{s}}_i = g^{-1} \dot{g} \dot{ \bm{s} }_i$ is calculated.
Also, \texttt{pim}, i.e., $\Pi$ is calculated according to $\eqref{eq:pim&frr}$.
Then $\ddot{h}$ is calculated according to $\eqref{eq:hdd}$.

If \texttt{calc} flag is set to \texttt{nd} or \texttt{nm}
then \texttt{SIGP} is called, if it is set to \texttt{sd} or
\texttt{sm} then \texttt{SIGS} is called, see \ref{sssec:sigs&p}
for more detail.
With
$\sigma_0 =
\{
\bm{a}_0 \times \bm{b}_0, \bm{b}_0 \times \bm{c}_0,
\bm{c}_0 \times \bm{a}_0
\}
= \frac{ V_0 }{ 2\pi } \{
\bm{c}^\ast_0, \bm{a}^\ast_0, \bm{b}^\ast_0
\}$,
we know
$\ddot{d} = \frac{ 1 }{ V_0 }\ddot{h} \sigma_0$.
Then we do strain symmetrization, $\ddot{d}_{ij} = \frac{ 1 }{ 2 }
(\ddot{d}_{ij} + \ddot{d}_{ji})$.
Then $\ddot{h} = \ddot{d} h_0$.

If in \texttt{nd} or \texttt{nm}, we do
$\Tr (\dot{h}\tran \sigma \sigma \dot{h})$.

If in \texttt{sd} or \texttt{sm}, we do
$\Tr(\dot{h}\tran \sigma_0 \sigma_0\tran \dot{ h })$.

Then in both cases, we do
\texttt{ekl} $=$ \texttt{ekl} + $\Tr (\dot{ h }\tran \dot{ h })$.

Total kinetic energy is $E_{t} = $ \texttt{eka} + $\frac{ 1 }{ 2 } W ekl$,
where $W$ is the fictitious mass,
total potential energy $U_{t} = ???E_{t} + P \Omega$, where $P$ is the external pressure.
Total energy is $\mathscr{E}_{t} = E_t + U_{t}$.

\begin{table}[h]
 \centering
 \caption{List of some variables in \texttt{INIT} subroutine.}
 \begin{tabular}{@{}lcr@{}}
  \toprule
  {variable name}         & variable symbol & variable        \\
  \midrule
  $\Pi$ matrix            & $\Pi$           & \texttt{pim}    \\
  forces on basis vectors & $\ddot{h}$      & \texttt{avec2d} \\
  pressure                & $P$             & \texttt{press}  \\
  \bottomrule
 \end{tabular}%
 \label{tab:init}%
\end{table}%
