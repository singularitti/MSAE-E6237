% !TEX root = ./report.tex

\subsubsection{\texttt{SIGS} and \texttt{SIGP} subroutine}
\label{sssec:sigs&p}

\texttt{SIGP} subroutine is used to calculate lattice vectors accelerations
based on ``new cell dynamics'', i.e., according to $\eqref{eq:rpsdd}$ and
$\eqref{eq:wenzhdd2}$; while \texttt{SIGS} is based on ``strain dynamics'',
i.e., $\eqref{eq:rpsdd}$ and $\eqref{eq:wenzhdd}$.
At last, the result returned is $\ddot{h}$.

In \texttt{SIGS} subroutine,
firstly calculates $f_0^{-1}$. By definition it is
\begin{equation}
	f_0^{-1} = \frac{ h_0 \tran h_0}{ V_0^2 },
\end{equation}
then set an argument \texttt{avint} to temporally store $\ddot{h}$,
and perform calculation $\ddot{h} = \ddot{h} f_0^{-1}$.
This subroutine also returns $\ddot{h}$ in the end.

In \texttt{SIGP} subroutine,
at first, calculates $f^{-1}$. As we know it is
\begin{equation}
	f^{-1} = \frac{ h \tran h }{ V^2 },
\end{equation}
and $e = \dot{ h }\tran \dot{ h }$, as stated above, as well as
a $3\times 3 \times 3 \times 3$ tensor
\begin{equation}
	f ' = \frac{ \partial f }{ \partial h_{ij} } = (\sigma'_{ij})\tran \sigma
	+ \sigma \tran \sigma'_{ij},
\end{equation}
where $\sigma'$ is denoted as \texttt{sigmap} in code, another $81 = 3\times 3
\times 3 \times 3$ tensor.
$\dot{f}_0 = \dot{ \sigma } \tran \sigma + \sigma \tran \dot{ \sigma }$ is
also calculated.
$f^{-1} = \sum_{k, l} e_{lk} f'_{ijkl}$, and $\sigma^{-1} = h \dot{ f }$.
Then final returns $\ddot{h}$. As stated above, $h = (1 + \epsilon) h_0$,
where $h_0 = \{ \bm{a}_0, \bm{b}_0, \bm{c}_0 \}$.
