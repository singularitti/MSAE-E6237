% !TEX root = ./report.tex

\section{Literature review}

\subsection{Invariant molecular dynamics with variable cell shape}

Here we introduce some key papers to trace the development of invariant molecular dynamics,
a.k.a., variable cell shape molecular dynamics.

\subsubsection{Andersen's approach}

Andersen first extends the molecular dynamics field to
ensembles other than micro-canonical ensemble.
In his ground-breaking paper (\cite{Andersen:1980ew}),
he proposed ways to calculate properties average
over isoenthalpic-isobaric (NPH) ensemble. He introduced
a Lagrangian
\begin{equation}\label{eq:andersenlagrang}
	\mathcal{L}(\bm{\rho}, \dot{\bm{\rho}}, Q, \dot{ Q }) = \frac{ 1 }{ 2 } m
	Q^{\frac{ 2 }{ 3 }}
	\sum_{i=1}^{N} \dot{ \bm{\rho} }_i \cdot \dot{ \bm{\rho} }_i - \sum_{i<j=1}^{N}
	u \big(Q^{\frac{ 1 }{ 3 }} \rho_{ij} \big) + \frac{ 1 }{ 2 } W \dot{ Q } ^2 -
	\alpha Q,
\end{equation}
where $\bm{\rho}_i = \bm{r}_i / V ^{\frac{ 1 }{ 3 }}$, $i=1$, $2$, $\ldots$,~$N$,
a dimensionless vector,
is called reduced coordinates of $i$th atom. $m$ is the atomic mass.
$Q^{\frac{ 1 }{ 3 }}\rho_{ij} = r_{ij}$ is the nearest distance between
$i$th and $j$th atoms.
Here $\alpha$ and $M$ are constants,
$\frac{ 1 }{ 2 } W \dot{ Q }^2$ now is regarded as a kinetic energy with fictitious
mass $W$,
and $\alpha Q$ is regarded as a potential energy for the motion of $Q$,
where $Q$ can be regarded as ``volume'', $\alpha$ is the ``pressure''.

The generalized momentum conjugate to $\bm{\rho}_i$ is
\begin{equation}
	\bm{\pi}_i = \frac{ \partial \mathcal{L} }{ \partial \dot{ \bm{\rho} }_i } =
	m Q^{\frac{ 2 }{ 3 }} \dot{\bm{\rho}}_i \thinspace,
\end{equation}
and which for $Q$ is
\begin{equation}
	\Pi = \frac{ \partial \mathcal{L} }{ \partial \dot{Q} } = M \dot{ Q }.
\end{equation}
The Hamiltonian is thus
\begin{equation}
	\begin{split}
		\mathcal{H}(\bm{\rho}, \bm{\pi}, Q, \Pi) &= \sum_{i=1}^{N} \bm{\pi}_i \cdot
		\dot{ \bm{\rho} }_i + \Pi \dot{ Q } - \mathcal{L}
		(\bm{\rho}, \dot{\bm{\rho}}, Q, \dot{ Q })\\
		&= \frac{ 1 }{ 2 m Q^{\frac{ 2 }{ 3 }} }
		\sum_{i=1}^{N} \bm{\pi}_i \cdot \bm{\pi}_i
		+ \sum_{i<j=1}^{N} u\big(Q^{\frac{ 1 }{ 3 }} \rho_{ij}\big) + \frac{ 1 }{ 2 M }
		\Pi^2 + \alpha Q.
	\end{split}
\end{equation}
So the equations of motions are
\begin{align}
	\dot{ \bm{\rho} }_i & = \frac{ \partial \mathcal{H} }{ \partial \bm{\pi}_i } =
	\frac{ \bm{\pi}_i }{ m Q^{\frac{ 2 }{ 3 }} } \thinspace,                          \\
	\dot{ \bm{\pi} }_i  & = - \frac{ \partial \mathcal{H} }{ \partial \bm{\rho}_i } =
	- Q^{\frac{ 1 }{ 3 }} \sum_{\substack{j=1                                         \\j\neq i}}
	\frac{ u' \bm{\rho}_{ij} }{ \lvert \bm{\rho}_{ij} \rvert  } \thinspace,           \\
	\dot{ Q }           & = \frac{ \partial \mathcal{H} }{ \partial \Pi } =
	\frac{ \Pi }{ W } \thinspace,                                                     \\
	\dot{ \Pi }         & = - \frac{ \partial \mathcal{H} }{ \partial Q } =
	- \frac{ 1 }{ 3Q } \bigg(
	- \frac{ 1 }{ m Q^{\frac{ 2 }{ 3 }} }	\sum_{i=1}^{N} \bm{\pi}_i \cdot \bm{\pi}_i
	+ Q^{\frac{ 1 }{ 3 }} \sum_{i<j} \rho_{ij} u'\big(Q^{\frac{ 1 }{ 3 } \rho_{ij}}\big) +
	3 \alpha Q
	\bigg) \thinspace.
\end{align}
With these equations, the trajectory of the reduced system is given by
$\bm{\rho}(t)$, $\bm{\pi}(t)$, $Q(t)$, and $\Pi(t)$.

The reduced system has a one-to-one correspondence to the original system, by
\begin{subequations}\label{eq:corres}
	\begin{align}
		V        & = Q,                                           \\
		\bm{r}_i & = Q^{\frac{ 1 }{ 3 }} \bm{\rho}_i \thinspace,  \\
		\bm{p}_i & = \bm{\pi}_i / Q^{\frac{ 1 }{ 3 }} \thinspace,
	\end{align}
\end{subequations}
to the phase space of a system spanned by
$\bm{r}_i, i=1$,~$\ldots$,~$N$ and
$\bm{p}_i, i=1$,~$\ldots$,~$N$,
where $\bm{r}_i$ is the atomic coordinate, $\bm{p}_i$ is its momentum.
Thus the trajectory
$\bm{\rho}(t)$, $\bm{\pi}(t)$, $Q(t)$, and $\Pi(t)$ have its
correspondence $V(t)$, $\bm{r}_i(t)$ and $\bm{p}_i(t)$ defined by
\eqref{eq:corres}.

Now suppose we have a function $\mathscr{G} = \mathscr{G}\big(\bm{r}(t), \bm{p}(t), V(t)\big)$,
and we want to know its time-average
\begin{equation}
	\overline{\mathscr{G}} =
	\lim_{T \rightarrow \infty} \frac{ 1 }{ T } \int_{0}^{T}  dt\,
	\mathscr{G}\big( \bm{r}(t), \bm{p}(t), V(t) \big).
\end{equation}

Since we have already found a one-to-one correspondence
from $\mathscr{G}$ to $G$ by
\eqref{eq:corres}, so they must have the same time-average
\begin{equation}\label{eq:bbgg}
	\overline{\mathscr{G}} = \overline{G}.
\end{equation}
$\overline{G}$ is easy to solve if we know
that using the trajectory
$(\bm{\rho}(t), \bm{\pi}(t), Q(t), \Pi(t))$,
any function's time average is given by
\begin{equation}
	\overline{G} = \lim_{T \rightarrow \infty} \frac{ 1 }{ T } \int_{0}^{T}  dt \,
	G(\bm{\rho}(t), \bm{\pi}(t), Q(t), \Pi(t)),
\end{equation}
and this is equal to the average of an NE ensemble,
\begin{equation}\label{eq:gbgne}
	\overline{G} = G_{NE},
\end{equation}
where
\begin{multline}
	G_{NE} (N, E) = \frac{ 1 }{ N! \Omega(N,E) } \int d\bm{\rho} \int d\bm{\pi}
	\int dQ \int d\Pi \\
	\delta \big( \mathcal{H}(\bm{\rho}, \bm{\pi}, Q, \Pi)
	- E \big) G(\bm{\rho}(t), \bm{\pi}(t), Q(t), \Pi(t)),
\end{multline}
with
\begin{equation}
	\Omega(N, E) = \frac{ 1 }{ N! }  \int d\bm{\rho} \int d\bm{\pi}
	\int dQ \int d\Pi \, \delta \big( \mathcal{H}(\bm{\rho}, \bm{\pi}, Q, \Pi)
	- E \big).
\end{equation}
Here we do not specify $V$ (or $Q$ here) in $G_{NE}$ because in this reduced system
$(\bm{\rho}(t), \bm{\pi}(t), Q(t), \Pi(t))$, $V$ is always $1$.

Thus from \eqref{eq:bbgg} and \eqref{eq:gbgne}, we have
\begin{equation}\label{eq:ggne}
	\overline{\mathscr{G}} = G_{NE},
\end{equation}
so what is $G_{NE}$ again?
It is not hard to find that
\begin{equation}\label{eq:gnegnph}
	G_{NE} = \mathscr{G}_{NPH}(N, \alpha, H),
\end{equation}
where $_{NPH}$ denotes the $NPH$ ensemble average, defined as
\begin{equation}
	\mathscr{G}_{NPH} (N, P, H) = \frac{ 1 }{ N!\Omega(N, P, H) }
	\int_{0}^{\infty} dV \int_{V} d\bm{r} \int d\bm{p}\,
	\delta(\mathcal{H}_{\bm{r}, \bm{p}, V} + P V - H)
	\mathscr{G}(\bm{r}, \bm{p}, V),
\end{equation}
where
\begin{equation}
	\Omega (N, P, H) = \frac{ 1 }{ N! }
	\int_{0}^{\infty} dV \int_{V} d\bm{r} \int d\bm{p}\,
	\delta(\mathcal{H}_{\bm{r}, \bm{p}, V} + P V - H) .
\end{equation}

So the final conclusion, by combining \eqref{eq:ggne} and \eqref{eq:gnegnph},
\begin{equation}
	\bar{\mathscr{G}} = \mathscr{G}_{NPH}(N, \alpha, H).
\end{equation}
Enthalpy $H$ for the ensemble is the energy of the trajectory of the reduced system,
minus the time average kinetic energy associated with the motion of $Q$.


\subsubsection{Rahman--Parrinello approach}
\label{sssec:rpa}

Rahman and Parrinello then proposed a method to perform
MD simulations which allows volume and shape of the MD cell
to change with time (\cite{Parrinello:1980kx}).
If the MD cell edges are $\bm{a}$, $\bm{b}$
and $\bm{c}$, which are time dependent. Stack them to form a
matrix $h = \{ \bm{a}, \bm{b}, \bm{c} \}$, and the volume of a cell
is then $\Omega = \bm{a} \cdot \bm{b} \times \bm{c}$, the metric
tensor is $g = h\tran h$. Let $i$th atomic coordinate be
$\bm{r}_i = \xi_i \bm{a} + \eta_i \bm{b}
	+ \zeta_i \bm{c} = h \bm{s}_i$, where $\bm{s}_i$ store its coordinates
$\xi_i$, $\eta_i$, and $\zeta_i$, i.e., reduced coordinates mentioned below (thus
$\xi$, $\eta$ and $\zeta$ all range from $0$ to $1$).
Then they introduced a Lagrangian, again for a periodic boundary condition,
\begin{equation}\label{eq:rplag}
	\mathcal{L} = \frac{ 1 }{ 2 } \sum_{i=1}^{N} m_i \dot{ \bm{s} }_i \tran
	g \dot{ \bm{s} }_i - \sum_{i=1}^{N} \sum_{i<j} \phi (r_{ij}) +
	\frac{ 1 }{ 2 } W \Tr \big(\dot{h} \tran \dot{h} \big) - P \Omega,
\end{equation}
where $r_{ij}^2 = (\bm{s}_i - \bm{s}_j)\tran g (\bm{s}_i - \bm{s}_j)$,
$P$ is the external hydrostatic pressure, $\phi(r_{ij})$ is the pair
potential, $W$ is the fictitious mass.

Following the same derivation steps as in Andersen's method, the equations of motion are
\begin{subequations}
	\label{eq:rpeqm}
	\begin{align}
		\ddot{s}_i & = \frac{ 1 }{ m_i } \sum_{j\neq i}
		\frac{ \phi'(r_{ij}) }{ r_{ij} } (\bm{s}_i - \bm{s}_j) - g^{-1}\dot{g}
		\dot{ \bm{s} }_i \thinspace, \quad i, j = 1, 2, \ldots, N,\label{eq:rpsdd} \\
		\ddot{h}   & = \frac{ 1 }{ W } (\Pi - P) \sigma, \label{eq:hdd}
	\end{align}
\end{subequations}
where $\sigma = \{\bm{a}\times \bm{b},
	\bm{b}\times \bm{c}, \bm{c}\times \bm{a}\}$,
matrix $\Pi$ is given by
\begin{equation}\label{eq:pim&frr}
	\Omega \Pi = \sum_{i=1}^{N} m_i \bm{v}_i \bm{v}_i\tran
	+ \sum_{i=1}^{N} \sum_{i<j} \frac{ \phi'(r_{ij}) }{ r_{ij} }
	(\bm{r}_i - \bm{r}_j) (\bm{r}_i - \bm{r}_j)\tran,
\end{equation}
with $\bm{v}_i = h \dot{ \bm{s}_i }$.
Then Andersen's equations of motion is a special case of
$\eqref{eq:rpeqm}$,
where
\begin{subequations}
	\begin{align}
		h              & = \diag \Big(  \Omega^{\frac{ 1 }{ 3 }}, \ldots,
		\Omega^{\frac{ 1 }{ 3 }}  \Big),                                  \\
		g^{-1} \dot{g} & =
		\frac{ 2 \dot{\Omega} }{3 \Omega }.
	\end{align}
\end{subequations}
Though his equation for
$\ddot{V}$ cannot be obtained from $\eqref{eq:hdd}$. But
this Lagrangian also results in a isoenthalpic, isobaric ensemble,
though with a small correction from the third term.


\subsubsection{Wentzcovitch's approach}

Wentzcovitch found that Rahman--Parrinello method is dependent on
the choice of cell edges (\cite{Wentzcovitch:1991ka}).
For different $\bm{a}$, $\bm{b}$, and
$\bm{c}$, the fictitious kinetic energy $K_L$ term could be different.
This will thus results in different trajectories even for the same system,
only with distinct basis choices.
For MD simulation of about $10^2$ particles the problem may
not be too serious. But if the cell experience a modular transformation,
the nominal value of $K_L$, and the resulted forces and trajectories
will be dependent on the choice of $h$. She then proposed a new Lagrangian
\begin{equation}\label{eq:wenzlag}
	\mathcal{L} = \sum_{i=1}^{N} \frac{ m_i }{ 2 }\dot{ \bm{q} }_i\tran d \dot{ \bm{q} }_i
	- \sum_{i=1}^{N} \sum_{i<j}^{N} \phi(r_{ij}) + \frac{ W }{ 2 }
	\Tr \big( \dot{ \epsilon } \dot{ \epsilon }\tran \big) - P \Omega,
\end{equation}
where $\epsilon$ is the strain, $\bm{q}_i$ is the defined as
$\bm{r}_i = (1+\epsilon) \bm{q}_i$, thus $d = (1 + \epsilon)\tran
	(1+\epsilon)$.
Similar derivation as above, the equations of motion are
\begin{subequations}
	\begin{align}
		\ddot{q}_i      & = - \frac{ 1 }{ m_i } \sum_{\substack{i, j=1 \\j\neq i}}^{N}
		\frac{ \phi'(r_{ij}) }{ r_{ij} }(\bm{q}_i - \bm{q}_j) - d^{-1} \dot{ d }
		\dot{ \bm{q} }_i \thinspace,                                   \\
		\ddot{\epsilon} & = \frac{ \Omega }{ W } (\Pi - P)
		\big((1+\epsilon)\tran\big)^{-1}.\label{eq:edd}
	\end{align}
\end{subequations}
An equivalent form of $\eqref{eq:wenzlag}$ is
\begin{equation}
	\mathcal{L} = \frac{ 1 }{ 2 } \sum_{i=1}^{N} m_i \dot{ \bm{s} }_i \tran
	g \dot{ \bm{s} }_i - \sum_{i=1}^{N} \sum_{i<j} \phi (r_{ij}) +
	\frac{ 1 }{ 2 } W \Tr \big(\dot{h} f_0 \dot{h}\tran \big) - P \Omega,
\end{equation}
with $f_0 = \sigma_0 \tran \sigma_0$, where $\sigma_0 = \{
	\bm{a}_0 \times \bm{b}_0, \bm{b}_0 \times \bm{c}_0,
	\bm{c}_0 \times \bm{a}_0 \}$. These vectors form the matrix $h_0$,
and the $h$ in $\eqref{eq:rplag}$ is $h = (1+\epsilon) h_0$.
This will lead to $\eqref{eq:edd}$ to be restated by using $\ddot{h}$
as
\begin{equation}\label{eq:wenzhdd}
	\ddot{h} = \frac{ 1 }{ W } (\Pi - P) \sigma f_0^{-1},
\end{equation}
which makes it easier to implement in code.

If further modify the fictitious kinetic energy term $K_L$ to be
$ \frac{ 1 }{ 2 } W \Tr \big( \dot{ h } \sigma\tran \sigma \dot{ h }\tran \big)$,
then $\eqref{eq:wenzhdd}$ will become
\begin{equation}\label{eq:wenzhdd2}
	\ddot{h} = \frac{ 1 }{ W } (\Pi - P) \sigma f^{-1} + \frac{ 1 }{ 2 }
	\Tr \bigg( e \frac{ \partial f }{ \partial h } \bigg) f^{-1}
	- \dot{h} \dot{ f } f^{-1},
\end{equation}
where $f = \sigma \tran \sigma$, $e = \dot{ h }\tran \dot{ h }$. This
Lagrangian coincide with Andersen's $\eqref{eq:andersenlagrang}$ in
isoshape limit.

She then performed MD simulations using $\eqref{eq:rplag}$ and
$\eqref{eq:wenzlag}$ on HCP-to-BCC
pressure-induced transformation in \ce{Mg} (\cite{Wentzcovitch:1994kc}),
as well as the modified Lagrangian under zero
temperature to compare. Her method eliminated the symmetry breaking
introduced by Rahman--Parrinello non-invariant fictitious part of the dynamics.

In summary, Andersen's approach allows us to do molecular dynamics with constant
pressure (variable cell volume) and cell shape. Rahman--Parrinello approach allows us to change the cell shape
and volume. Wentzcovitch found that Rahman--Parrinello's Lagrangian is dependent
on the choice of cell edges, so she extended this to a method independent on
the cell edges' choice, which simplified the calculation procedure.
