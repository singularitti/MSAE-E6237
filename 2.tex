% !TEX root = report.tex

\section{Invariant molecular dynamics with variable cell shape}

\subsection{Andersen's approach}

It was Anderson who first extends the molecular dynamics field to
ensembles other than micro-canonical ensemble.\cite{Andersen:1980ew}
In his ground-breaking paper, he proposed ways to calculate properties average
over isoenthalpic-isobaric (NPH) ensemble. In the article, he introduced
a Lagrangian
\begin{equation}\label{eq:lagrang}
	\mathcal{L}(\bm{\rho}, \dot{\bm{\rho}}, Q, \dot{ Q }) = \frac{ 1 }{ 2 } m
	Q^{\frac{ 2 }{ 3 }}
	\sum_{i=1}^{N} \dot{ \bm{\rho} } \cdot \dot{ \bm{\rho} } - \sum_{i<j=1}^{N}
	u \big(Q^{\frac{ 1 }{ 3 }} \rho_{ij} \big) + \frac{ 1 }{ 2 } M \dot{ Q } ^2 -
	\alpha Q,
\end{equation}
where $\bm{\rho}_i = \bm{r}_i / V ^{\frac{ 1 }{ 3 }}$, $i=1$, $2$, $\ldots$,~$N$,
is called scaled coordinates. Here $\alpha$ and $M$ are constants,
$\frac{ 1 }{ 2 } M \dot{ Q }$ now is regarded as a kinetic energy with fictitious
mass $M$,
and $\alpha Q$ is regarded as a potential energy for the motion of $Q$.
The generalized momentum conjugate to $\bm{\rho}$ is
\begin{equation}
	\bm{\pi}_i = \frac{ \partial \mathcal{L}_2 }{ \partial \dot{ \bm{\rho} }_i } =
	m Q^{\frac{ 2 }{ 3 }} \bm{\rho}_i,
\end{equation}
and which for $Q$ is
\begin{equation}
	\Pi = \frac{ \partial \mathcal{L}_2 }{ \partial \dot{Q} } = M \dot{ Q }.
\end{equation}
The Hamiltonian is thus
\begin{equation}
	\begin{split}
		\mathcal{H}(\bm{\rho}, \bm{\pi}, Q, \Pi) &= \sum_{i=1}^{N} \bm{\pi}_i \cdot
		\dot{ \bm{\rho} }_i + \Pi \dot{ Q } - \mathcal{L}_2
		(\bm{\rho}, \dot{\bm{\rho}}, Q, \dot{ Q })\\
		&= \frac{ 1 }{ 2 m Q^{\frac{ 2 }{ 3 }} }
		\sum_{i=1}^{N} \bm{\pi}_i \cdot \bm{\pi}_i
		+ \sum_{i<j=1}^{N} u\big(Q^{\frac{ 1 }{ 3 }} \rho_{ij}\big) + \frac{ 1 }{ 2 M }
		\Pi^2 + \alpha Q.
	\end{split}
\end{equation}
So the equations of motions are
\begin{align}
	\dot{ \bm{\rho} }_i & = \frac{ \partial \mathcal{H} }{ \partial \bm{\pi}_i } =
	\frac{ \bm{\pi}_i }{ m Q^{\frac{ 2 }{ 3 }} }\\
	\dot{ \bm{\pi} }_i  & = - \frac{ \partial \mathcal{H} }{ \partial \bm{\rho}_i } =
	- Q^{\frac{ 1 }{ 3 }} \sum_{\substack{j=1\\j\neq i}}
	\frac{ u' \bm{\rho}_{ij} }{ \lvert \bm{\rho}_{ij} \rvert  }\\
	\dot{ Q }           & = \frac{ \partial \mathcal{H} }{ \partial \Pi } =
	\frac{ \Pi }{ M } \\
	\dot{ \Pi }         & = - \frac{ \partial \mathcal{H} }{ \partial Q } =
	- \frac{ 1 }{ 3Q } \bigg(
	- \frac{ 1 }{ m Q^{\frac{ 2 }{ 3 }} }	\sum_{i=1}^{N} \bm{\pi}_i \cdot \bm{\pi}_i
	+ Q^{\frac{ 1 }{ 3 }} \sum_{i<j} \rho_{ij} u'\big(Q^{\frac{ 1 }{ 3 } \rho_{ij}}\big) +
	3 \alpha Q
	\bigg)
\end{align}
With these equations, the trajectory of the scaled system are given by
$\bm{\rho}(t)$, $\bm{\pi}(t)$, $Q(t)$, and $\Pi(t)$.

Use this trajectory, any function's time average are given by
\begin{equation}
	\overline{G} = \lim_{T \rightarrow \infty} \frac{ 1 }{ T } \int_{0}^{T}  \, dt
	G(\bm{\rho}(t), \bm{\pi}(t), Q(t), \Pi(t)),
\end{equation}
and this can be given by the average of an NE ensemble. That is,
\begin{multline}
	G_{NE} (N, E) = \frac{ 1 }{ N! \Omega(N,E) } \int d\bm{\rho} \int d\bm{\pi}
	\int dQ \int d\Pi \\
	\delta \big( \mathcal{H}(\bm{\rho}, \bm{\pi}, Q, \Pi)
	- E \big) G(\bm{\rho}(t), \bm{\pi}(t), Q(t), \Pi(t)),
\end{multline}
where
\begin{equation}
	\Omega(N, E) = \frac{ 1 }{ N! }  \int d\bm{\rho} \int d\bm{\pi}
	\int dQ \int d\Pi \, \delta \big( \mathcal{H}(\bm{\rho}, \bm{\pi}, Q, \Pi)
	- E \big).
\end{equation}

The scaled system has a correspondence
\begin{align}\label{eq:corres}
	V        & = Q,                               \\
	\bm{r}_i & = Q^{\frac{ 1 }{ 3 }} \bm{\rho}_i, \\
	\bm{p}_i & = \bm{\pi}_i / Q^{\frac{ 1 }{ 3 }}
\end{align}
to the phase space of a system spanned by $\{ \bm{r}_i \}$ and
$\{ \bm{p}_i \}$. Thus the trajectory
$\bm{\rho}(t)$, $\bm{\pi}(t)$, $Q(t)$, and $\Pi(t)$ have its
correspondence $V(t)$, $\bm{r}_i(t)$ and $\bm{p}_i(t)$ by
$\eqref{eq:corres}$.
Since the time average $\overline{F}$ of any function $F$
derived by this trajectory is the same as an
isoentahlpic-isobaric ensemble average of $F_{NPH}$, and
$\overline{G} = \overline{F}$, thus we get the NPH ensemble average.
The ensemble pressure $P$ is $\alpha$ in $\eqref{eq:lagrang}$ indeed.


\subsection{Rahman and Parrinello approach}

