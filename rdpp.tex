% !TEX root = ./report.tex

\subsubsection{\texttt{RDPP} subroutine}

This subroutine reads the pair-potential file.
Here \texttt{ntype} is the number of different types of atoms.
It reads the potential of $i$th atom and $j$th atom, where $j \geq i$.
In all, $n (n+1)$ pair potentials are read, where $n$ denotes variable \texttt{ntype}.
So in the code we need to do some assignments like
\begin{align}
	U_{ij}      & = U_{ji},      \\
	\bm{F}_{ij} & = \bm{F}_{ji},
\end{align}
where $U_{ji}$ and $\bm{F}_{ji}$ are what we read from file.
Thus we can save time on IO operations because the pair-potentials are symmetric.
