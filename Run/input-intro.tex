% !TEX root = report.tex

\subsection{Input file}

Now let's have a look at the input file.

First we see the first input file \texttt{inp1}, with $4$ \ce{Ar} atoms in a FCC cell,
the atom mass is assumed to be \SI{40}{\atomicmassunit},
see Fig. \ref{fig:fcc1} for reference.
\begin{figure}[h]
 \centering
 \input{Run/Tikz/FCC1}
 \caption{A FCC cell of \ce{Ar} for \texttt{inp1}.}
 \label{fig:fcc1}
\end{figure}

Run the simulation code, we derive the following results,
see Fig. \ref{fig:input1} for reference.
\begin{figure}[h]
 \centering
 \begin{minipage}[t]{0.45\textwidth}
  \includegraphics[width=\linewidth]{input1/avec_abc}
  \subcaption{Lattice parameters of \texttt{inp1}. The dashed dot lines are upper and lower
   bound of the lattice parameters. We can see that $a$, $b$, $c$ lines coincide with each
  other.}
  \label{fig:input1:ee}
 \end{minipage}
 \hfil
 \begin{minipage}[t]{0.45\textwidth}
  \includegraphics[width=\linewidth]{input1/t}
  \subcaption{Total energy of \texttt{inp1}. The dashed dot lines are upper and lower
   bound of the total kinetic energy and total potential energy. Total energy is the sum
  of the other two.}
  \label{fig:input1:e}
 \end{minipage}
 \hfil
 \vfill
 \begin{minipage}[t]{0.45\textwidth}
  \includegraphics[width=\linewidth]{input1/a}
  \subcaption{Atomic contribution to total energy of \texttt{inp1}.
   The dashed dot lines are upper and lower bound of the
   total atomic energy. We can see that atoms do not contribute
  to total kinetic energy.}
  \label{fig:input1:a}
 \end{minipage}
 \hfil
 \begin{minipage}[t]{0.45\textwidth}
  \includegraphics[width=\linewidth]{input1/l}
  \subcaption{Lattice contribution to total energy of \texttt{inp1}.
   The dashed dot lines are upper and lower bound of the
   total lattice energy. We can see that lattice do not contribute
  to total potential energy.}
  \label{fig:input1:l}
 \end{minipage}
 \caption{Simulation results for \texttt{inp1}.}
 \label{fig:input1}
\end{figure}
