% !TEX root = report.tex

\section{First principles molecular dynamics}

The first quantum molecular dynamics (QMD),
a.k.a., ``\textit{ab initio}'' or ``first principles'' simulation was
carried out by Car and Parrinello in 1985.\cite{Car:1985ix}
Their and subsequent work
have led to a great development on treating real, complex molecules,
solids and liquids with forces derived by density functional theory
(DFT).\cite{martin2004electronic} Now we are going to have a look
on this work.


\subsection{Car--Parrinello approach}

Car and Parrinello were dedicating to find the ground-state electronic
solution for the electrons as the nuclei move. Their MD simulation was
based on $2$ assumptions: the validity of classical mechanics to describe
ionic motions, and Born--Oppenheimer approximation to separate the
nuclear and electronic coordinates.
They invented a new strategy other than the Metropolis Monte Carlo method
introduced by Kirkpatrick, Gelatt, and Vecchi,\cite{Kirkpatrick:1983zz}
the so-called ``dynamical simulated annealing'' method.
By this method, they could minimize the energy of electrons and solve
for the motion of the nuclei simultaneously. This is achieved by introducing
a Lagrangian
\begin{equation}\label{eq:cplag}
	\mathcal{L} = \sum_{i} \frac{ 1 }{ 2 } \mu \int d\bm{r} \lvert
	\dot{\psi}_i(\bm{r}) \rvert ^ 2 +
	\sum_{I} \frac{ 1 }{ 2 } M_I \dot{ \bm{R} }_I ^ 2 +
	\sum_{\nu} \frac{ 1 }{ 2 } \mu_{\nu}  \dot{ \alpha }_\nu ^ 2 -
	E[\{ \psi_i \}, \{ \bm{R}_I \}, \{ \alpha_\nu \}],
\end{equation}
with the holonomic constraints
\begin{equation}\label{eq:cpconst}
	\int d\bm{r} \psi_i ^ \ast (\bm{r}, t) \psi_j (\bm{r}, t) = \delta_{ij},
\end{equation}
where $M_I$ are the physical ionic masses, while $\mu$ and $\mu_{\nu}$
are just arbitrary parameters of appropriate units.
In $\eqref{eq:cplag}$,
the dynamics associated with the $\{ \psi_i \}$'s and
the $\{ \alpha_\nu \}$'s is fictitious and should only be considered
as a numerical tool.
Combine
$\eqref{eq:cplag}$ and $\eqref{eq:cpconst}$ we derive
\begin{multline}
	\mathcal{L}' =
	\sum_{i} \frac{ 1 }{ 2 } \mu \int d\bm{r} \lvert
	\dot{\psi}_i(\bm{r}) \rvert ^ 2 +
	\sum_{I} \frac{ 1 }{ 2 } M_I \dot{ \bm{R} }_I ^ 2 +
	\sum_{\nu} \frac{ 1 }{ 2 } \mu_{\nu}  \dot{ \alpha }_\nu ^ 2 -
	E[\{ \psi_i \}, \{ \bm{R}_I \}, \{ \alpha_\nu \}]\\
	+ \sum_{ij} \Lambda_{ij} \bigg[
		\int d\bm{r} \psi_i ^ \ast (\bm{r}, t) \psi_j (\bm{r}, t) - \delta_{ij}
		\bigg],
\end{multline}
where $E[\{ \psi_i \}, \{ \bm{R}_I \}, \{ \alpha_\nu \}]$ is a functional,
$\Lambda_{ij}$ is a Lagrangian multiplier.
Thus the corresponding equations of motion are
\begin{align}
	\mu \ddot{\psi}_i (\bm{r}, t) & =
	\frac{ \partial \mathcal{L}' }{ \partial \psi_i^\ast (\bm{r}, t)} = -
	\frac{ \delta E}{ \delta \psi_i^\ast (\bm{r}, t)} + \sum_{k}
	\Lambda_{ik} \psi_k (\bm{r}, t) =
	- \hat{H} \psi_i(\bm{r}, t) + \sum_{k}
	\Lambda_{ik} \psi_k (\bm{r}, t)\\
	M_I \ddot{\bm{R}}_I           & =
	\frac{ \partial \mathcal{L}' }{ \partial \bm{R}_I } =
	\bm{F}_{I} =
	- \frac{ \partial E }{ \partial \bm{R}_I } \thinspace,\\
	\mu_{\nu} \ddot{\alpha}_{\nu} & =
	\frac{ \partial \mathcal{L}' }{ \partial \alpha_{\nu} } =
	- \frac{ \partial E }{ \partial \alpha_{\nu} } \thinspace,
\end{align}
where $\hat{H}$ is the hamiltonian.
These equations can be simulated by the Verlet algorithm, that is,
\begin{subequations}
	\label{eq:cpeqm}
	\begin{align}
		\psi_i ^{n+1} (\bm{r}) & = 2 \psi_{i} ^ {n} (\bm{r}) -
		\psi_i ^{n-1} (\bm{r}) - \frac{ h^2 }{ \mu }
		\bigg[
			\hat{H} \psi_{i} ^ {n} (\bm{r}) - \sum_{k}
			\Lambda_{ik} \psi_k (\bm{r}, t)
			\bigg],\label{eq:cpst}\\
		\bm{R}_{I}^{n+1}       & = 2 \bm{R}_{I}^{n} - \bm{R}_{I}^{n-1} +
		\frac{ h^2 }{ M_I } \bm{F}_{I} \thinspace,\\
		\alpha_{\nu}^{n+1}     & = 2 \alpha_{\nu}^{n} -
		\alpha_{\nu}^{n-1} -
		\frac{ h^2 }{ \mu_{\nu} }
		\frac{ \partial E }{ \partial \alpha_{\nu} } \thinspace,
	\end{align}
\end{subequations}
where $h$ is the time step interval. The holonomic constraints are
handled by a method called SHAKE.\cite{Ryckaert:1977gp}

At stationary state, all time derivatives are $0$ so that
$\eqref{eq:cpst}$ leads to
\begin{equation}
	\hat{H} \psi_{i} ^ {n} (\bm{r}) = \sum_{k}
	\Lambda_{ik} \psi_k (\bm{r}, t),
\end{equation}
which shows that $\Lambda$ is the transpose of
usual Kohn--Sham hamiltonian matrix
$H$, i.e., $\Lambda_{ij} = H_{ji}$. Thus the eigenvalues of
$\Lambda$ coincides with Kohn--Sham eigenvalues.
And the solution is stationary if and only if the Kohn--Sham
energy is at a variational minimum.\cite{martin2004electronic}
This method leads to the possibility to consider real dynamics
of the nuclei in \textit{ab initio} electronic structure algorithms.


\subsection{Wentzcovitch's approach}

In Car--Parrinello method, as $\eqref{eq:cpeqm}$ has shown,
at each timestep $\hat{H} \psi_i$ is calculated. However, this is
the most time-consuming operation in the algorithm.




