% !TEX root = report.tex

\section{Molecular dynamics code}

The following part is an introduction on how to use
Wentzcovitch's code to perform MD simulation. The code is
based on one of her papers.\cite{Wentzcovitch:1991ka}

The simulation is done under zero temperature, as stated above.




\subsection{\texttt{UPDG} subroutine}
This part is used to update several quantities of cell during calculations.

Here \texttt{avec} is $h$, \texttt{avecd} is $\dot{ h }$, thus $g = h\tran h$ 
and $\dot{g}  = \dot{ h }\tran h + h \dot{ h }\tran$, \texttt{gm1} is $g^{-1}$
and \texttt{gmgd} is $g^{-1} \dot{g}$. \texttt{sigma} is the reciprocal lattice
vectors $\sigma$.

It first reads a flag \texttt{itg} to see if $\sigma$ and $V$ need to be calculated,
if it is `yes', then do the following things:
Firstly calculates $\sigma$ by the components of $h$, and then calculate the 
MD cell volume by
\begin{equation}
  V = \sigma \cdot h,
\end{equation}
and then calculate $g$, $\dot{ g }$, $g^{-1}$, $g^{-1}\dot{g}$, respectively.

\subsection{\texttt{SIGP} and \texttt{SIGS} subroutine}

\texttt{SIGP} subroutine is used to calculate lattice vectors accelerations 
based on `new dynamics', i.e., according to $\eqref{eq:rpsdd}$ and
$\eqref{eq:wenzhdd}$; while \texttt{SIGS} is based on `strain dynamics',
i.e., $\eqref{eq:rpsdd}$ and $\eqref{eq:wenzhdd2}$.

Firstly calculates $f^{-1}$, by definition we know it is
\begin{equation}
  f^{-1} = \frac{ h \tran h }{ V },
\end{equation}
and $e = \dot{ h }\tran \dot{ h }$, as stated above, as well as
a $3\times 3 \times 3 \times 3$ tensor
\begin{equation}
  f ' = \frac{ \partial f }{ \partial h_{ij} } = (\sigma'_{ij})\tran \sigma
  + \sigma \tran \sigma'_{ij},
\end{equation}
where $\sigma'$ is denoted as \texttt{sigmap} in code, another $3\times 3
\times 3 \times 3$ tensor.
$\dot{f}_0 = \dot{ \sigma } \tran \sigma + \sigma \tran \dot{ \sigma }$ is 
also calculated.
$f^{-1} = \sum_{k, l} e_{lk} f'_{ijkl}$, and $\sigma^{-1} = h \dot{ f }$.
Then final returns $\ddot{h}$.





\subsection{\texttt{RDPP} subroutine}