% !TEX root = ./report.tex

\section{Integration algorithms for molecular dynamics}
\label{sec:open}

Suppose we have a general second-order ODE, of the form
\begin{equation}
	x''(t) = F(x),
\end{equation}
and we want to solve it. The force $F$ can be given by differentiating Lagrangian or Hamiltonian, as we do in \eqref{eq:rpeqm}, etc. Two major types of methods are
available: \textit{open} and \textit{closed} methods.

Closed methods, or \textit{predictor-corrector} methods, firstly predict a value
$x_{n}'$ with a predicting formula and then use $F\big(x_{n}'\big)$ to get the $x_{n+1}$
for next time step. Closed methods with only one force evaluation, acceptable in time-step-integration,
are often called \textit{Gear} algorithm.

Open methods, or \textit{predictor} methods predict $x_{n+1}$ only depend on all
information we have no further than $n$th step. That is, there is no corrector $x_n'$ between
$x_{n}$ and $x_{n+1}$. In the following text some open methods will be introduced since
they are often used in molecular dynamics (MD).

The simplest  open method to solve the ODE numerically is Euler's method:
\begin{subequations}
	\begin{align}
		x_{n+1} & = x_n + h v_n + \frac{ 1 }{ 2 } h^2 F(x_n), \\
		v_{n+1} & = v_n + h F(x_n),
	\end{align}
\end{subequations}
where $h$ is the length of each time step. We are already familiar with this method.

The Verlet algorithm, another open method, starts from the basic equation
\begin{equation}\label{eq:verlet}
	\frac{ x_{n+1} + x_{n-1} - 2 x_n }{ h^2 } = F(x_n),
\end{equation}
which is, the second-order central difference method.
If we write $x_{n+1}$ and $x_{n-1}$ as
\begin{align}
	x_{n+1} & = x_n + h v_n + \frac{ 1 }{ 2 } h^2 x_n'' + \frac{ 1 }{ 6 } h^3 x_n''' + \mathcal{O}(h^4), \\
	x_{n-1} & = x_n - h v_n + \frac{ 1 }{ 2 } h^2 x_n'' - \frac{ 1 }{ 6 } h^3 x_n''' + \mathcal{O}(h^4),
\end{align}
thus up to third order, we have
\begin{equation}\label{eq:vverlet}
	\frac{x_{n+1} - x_{n-1}}{2h} =  v_n.
\end{equation}
So combine \eqref{eq:vverlet} and \eqref{eq:verlet} we obtain
\begin{equation}\label{eq:xverlet}
	x_{n+1} = x_{n} + h v_{n} + \frac{ 1 }{ 2 } h^2 F(x_n),
\end{equation}
we can see that this falls back to Euler's method. Equations
\eqref{eq:vverlet} and \eqref{eq:xverlet} defines the Verlet algorithm.

The leap-frog algorithm is usually described as
\begin{subequations}
	\begin{align}
		x_{n+1}   & = x_n + h v_{n+1/2}, \label{eq:lfx}   \\
		v_{n+1/2} & = v_{n-1/2} + h F(x_n).\label{eq:lfv}
	\end{align}
\end{subequations}
The derivation is still from Taylor expansion. First,
\begin{equation}
	x_{n+1} = x_{n} + h \bigg(v_n + \frac{ h }{ 2 } x_n'' \bigg) + \mathcal{O}(h^3),
\end{equation}
thus up to third order, we have
\begin{equation}
	v_{n + 1/2} = v_n + \frac{ h }{ 2 } x_n'' = v_n + \frac{ h }{ 2 } F(x_n),
\end{equation}
thus \eqref{eq:lfx} is derived.
Similarly $v_{n-1/2} = v_n - \frac{ h }{ 2 } F(x_n)$, so \eqref{eq:lfv} is derived.
If we assume $v_n = \frac{ 1 }{ 2 }(v_{n-1/2} + v_{n+1/2})$, thus
\begin{equation}\label{eq:lfv_n}
	v_n = \frac{ 1 }{ 2h }(x_{n+1} - \cancel{x_n} + \cancel{x_n} - x_{n-1}),
\end{equation}
so \eqref{eq:lfx} and \eqref{eq:lfv_n} are equivalent to \eqref{eq:vverlet} and \eqref{eq:xverlet}.

One of the most important is the Beeman algorithm, which is used in the
variable cell shape molecular dynamics (VCSMD) code.
Its basic equations are
\begin{subequations}\label{eq:beeman}
	\begin{align}
		x_{n+1} & = x_n + h v_n + \frac{ 2 }{ 3 } h^2 F_n - \frac{ 1 }{ 6 } h^2 F_{n-1} \thinspace,
		\label{eq:xbeeman}\\
		v_{n+1} & = v_n + \frac{ 1 }{ 3 }h F_{n+1} + \frac{ 5 }{ 6 } h F_n - \frac{ 1 }{ 6 } h F_{n-1}
		\thinspace,
		\label{eq:vbeeman}
	\end{align}
\end{subequations}
where $F_n = F(x_n)$ is a shorthand of the force on $n$th step (\cite{dufty1986molecular}). This algorithm
is equivalent to the Verlet algorithm. Let's prove it. First consider
\begin{equation}
	v_{n-1} = \frac{ 1 }{ h }(x_n - x_{n-1}) - \frac{ 1 }{ 6 } h (4 F_{n-1} - F_{n-2}),
\end{equation}
by transforming $\eqref{eq:xbeeman}$, then use $\eqref{eq:vbeeman}$ we derive
\begin{equation}\label{eq:vnverlet}
	v_n = \frac{ 1 }{ h }(x_n - x_{n-1}) + \frac{ 1 }{ 6 } h F_{n-2} + \frac{ 5 }{ 6 } h (F_n - F_{n-1})
	+ \frac{ 1 }{ 3 }h F_{n+1} \thinspace,
\end{equation}
substitute $\eqref{eq:vnverlet}$ back into $\eqref{eq:xbeeman}$ then $\eqref{eq:verlet}$ is recovered.
Beeman algorithm is more complex and memory-consuming, usually it is less likely to be used in
real program.

Above we have proved that the Verlet, leap-frog, and Beeman algorithm are equivalent,
so they are expected to produce the same trajectory.
These methods are time-reversible, so they essentially have no drift. They are often
served as integrators in molecular dynamics, about which we will talk later.